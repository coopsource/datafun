\documentclass[ribbons]{rntz}
\usepackage[libertine]{rntzfont}
\usepackage[a5paper,total={12cm,18cm}]{geometry}


%% ---- Packages ----
\usepackage{mathpartir}         % \mathpar, \infer
\usepackage{multirow}
\usepackage{stmaryrd}           % for semantic brackets
\usepackage{mathtools}          % for \dblcolon
\usepackage{tikz,tikz-cd}       % Hasse & commutative diagrams.
\usepackage{adjustbox}          % aligning tikz diagrams vertically w/ tables.
\usepackage{booktabs}           % \midrule
\usepackage{anyfontsize}        % avoid font size warnings from stmaryrd


%% ---- Commands ----
\newcommand{\todo}[1]{{\itshape\color{blue}#1}}

\newcommand{\bnfeq}{\dblcolon=}
%% \newcommand{\defeq}{\overset{\ms{def}}{=}}

\newcommand{\ms}[1]{\ensuremath{\mathsf{#1}}}
\newcommand{\mb}[1]{\ensuremath{\mathbf{#1}}}
\newcommand{\mi}[1]{\ensuremath{\mathit{#1}}}
\newcommand{\mc}[1]{\ensuremath{\mathcal{#1}}}

\newcommand{\GG}{\Gamma}
\newcommand{\N}{\mathbb{N}}
\newcommand{\x}{\times}
\newcommand{\fn}{\lambda}
\newcommand{\binder}{.\,}
\newcommand{\bind}[1]{{#1}\binder}
\newcommand{\fnof}[1]{\fn\bind{#1}}
\newcommand{\den}[1]{\llbracket{#1}\rrbracket}

%% Tones & tone operators.
\newcommand{\tone}{1}                    % used once
\newcommand{\tzero}{0}                   % used zero times
\newcommand{\taff}{\ms{aff}}             % used affinely (zero or one).
\newcommand{\tmany}{\ensuremath{\infty}} % used any number of times.

\newcommand{\tc}{\cdot}                 % tone composition
\newcommand{\tmul}{+}                   % tone addition/multiplication
\newcommand{\tmeet}{\wedge}             % tone meet


%% ---- Front matter ----
\title{Tones for linearity}
\author{Michael Arntzenius, %
  \href{mailto:daekharel@gmail.com}{daekharel@gmail.com}}
% Date format: "25 March 2018"
\usepackage[en-GB]{datetime2}
\DTMlangsetup[en-GB]{ord=omit}
%% date started: 2018-04-23
\date{\today}


\begin{document}
\maketitle

{\vspace{1em}%\small
\textbf{Abstract.}
%
 I present a sequent calculus for intuitionistic linear logic in which
 hypotheses are annotated with ``tones''. Surprisingly, this involves three
 monoids, rather than a semiring, as might be expected. \todo{TODO:
   Contextualize this; compare to Pfenning-style multiple contexts and
   McBride-and-other-folk's-style semiring annotations.} \par}

\section{A tonal sequent calculus for intuitionistic linear logic}

\newcommand{\with}{\ensuremath{\mathrel{\&}}}
\newcommand{\ox}{\ensuremath{\otimes}}
\newcommand{\lolli}{\ensuremath{\multimap}}
\newcommand{\bang}{{!}}

\subsection{Syntax}

\[
\begin{array}{cccl}
  \text{tones} & s &\bnfeq& \tzero ~|~ \tone ~|~ \taff ~|~ \tmany
  \\
  \text{types} & A,B &\bnfeq& \bang A ~|~ A \with A ~|~ A \ox A ~|~ A \lolli A
  \\
  \text{contexts} & \GG &\bnfeq& \varepsilon ~|~ \GG, A^s
  \\
  \text{judgments} & J &\bnfeq& \GG \vdash A
\end{array}
\]


\subsection{Tones}

Tones are annotations on hypotheses. In linear logic, we'll use tones to track
how many times a hypothesis is used in a proof:

\begin{center}
  \begin{tabular}{cl}
    \textit{Tone} & \textit{Usage}\\\midrule
    \tzero & Not at all.\\
    \tone  & Exactly once.\\
    \taff  & At most once (``affinely'').\\
    \tmany & Any number of times.
  \end{tabular}
\end{center}

Tones are partially ordered, $\tmany < \taff < \{\tzero, \tone\}$. I let $s \le
t$ iff permission to use a hypothesis $s$ times implies permission to use it
only $t$ times. For example, $\taff \le \tzero$, because if we are allowed to
use something \emph{at most once}, we may choose to use it \emph{not at all}.

\subsection{Tone operators}

We will need three operators on tones, defined in Figure~\ref{fig:tone-ops}:
meet $s \tmeet t$, addition $s \tmul t$, and composition $s \tc t$. Meet is the
greatest lower bound of two tones; \todo{TODO}.

\todo{Figure out which laws apply. I suspect $\tc$ distributes over both
  $\tmeet$ and $\tmul$. Do $\tmeet$ and $\tmul$ have any interesting laws?}

\begin{figure*}
  \begin{mathpar}
    \begin{tikzpicture}[scale=1,baseline=(current bounding box.center)]
      \node (0)    at (-1, 1) {$\tzero$};
      \node (1)    at ( 1, 1) {$\tone$};
      \node (aff)  at ( 0, 0) {$\taff$};
      \node (many) at ( 0,-1) {$\tmany$};
      \draw (0) -- (aff) -- (1);
      \draw (aff) -- (many);
    \end{tikzpicture}

    \begin{array}{r|cccc}
      s \tmul t & \tzero & \tone  & \taff  & \tmany\\\hline
      \tzero    & \tzero & \tone  & \taff  & \tmany\\
      \tone     & \tone  & \tmany & \tmany & \tmany\\
      \taff     & \taff  & \tmany & \tmany & \tmany\\
      \tmany    & \tmany & \tmany & \tmany & \tmany
    \end{array}

    \begin{array}{r|cccc}
      s \tc t & \tzero & \tone  & \taff  & \tmany\\\hline
      \tzero  & \tzero & \tzero & \tzero & \tzero\\
      \tone   & \tzero & \tone  & \taff  & \tmany\\
      \taff   & \tzero & \taff  & \taff  & \tmany\\
      \tmany  & \tzero & \tmany & \tmany & \tmany
    \end{array}
    \vspace{-1em}
  \end{mathpar}
  \caption{Tone meet-semilattice and operators}
  \label{fig:tone-ops}
\end{figure*}


\subsection{Extending tone operators to contexts}

\todo{TODO}


\subsection{Rules}

\begin{mathpar}
  \infer[{\bang} left]{\GG, A^{\tmany\tc s} \vdash C}{\GG, {\bang A}^s \vdash C}

  \infer[{\with} left$_i$]{\GG, A_i^s \vdash C}{\GG, (A_1 \with A_2)^s \vdash C}

  \infer[{\ox} left]{\GG, A^\tone, B^\tone \vdash C}
        {\GG, (A \ox B)^\tone \vdash C}

  \infer[{\lolli} left]{\GG_1 \vdash A \\ \GG_2, B^s \vdash C}
        {\GG_1^s \tmul \GG_2, (A \lolli B)^s \vdash C}
\\
  \infer[{\bang} rite]{\GG \vdash A}{\GG^\tmany \vdash \bang A}

  \infer[{\with} rite]{\GG_1 \vdash A \\ \GG_2 \vdash B}
        {\GG_1 \tmeet \GG_2 \vdash A \with B}

  \infer[{\ox} rite]{\GG_1 \vdash A \\ \GG_2 \vdash B}
        {\GG_1 \tmul \GG_2 \vdash A \ox B}

  \infer[{\lolli} rite]{\GG, A^\tone \vdash B}{\GG \vdash A \lolli B}
\\
  \infer[cut]{\GG_1 \vdash A \\ \GG_2, A^s \vdash B}
        {\GG_1^s \tmul \GG_2 \vdash B}
\end{mathpar}

\todo{TODO: \textsc{\ox-left} could be more general. Think about how to
  generalize it.}

\end{document}
