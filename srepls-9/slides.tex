\documentclass[xcolor={dvipsnames}]{beamer}

\usefonttheme{serif}
\usepackage[pteuler]{rntzfont} % pteuler, charter, cochineal work well
\usepackage{PTSans}
\usepackage[semibold]{sourcesanspro}

\usepackage{url,hyperref}
\usepackage{mathpartir}
\usepackage{latexsym,amssymb,stmaryrd,mathtools}
\usepackage{anyfontsize} % fix font warnings from stmaryrd
%\usepackage{booktabs}
\usepackage{multirow}
\usepackage[normalem]{ulem}               % for \sout
\newcommand{\msout}[1]{\text{\sout{\ensuremath{#1}}}}

\definecolor[named]{ACMBlue}{cmyk}{1,0.1,0,0.1}
\definecolor[named]{ACMYellow}{cmyk}{0,0.16,1,0}
\definecolor[named]{ACMOrange}{cmyk}{0,0.42,1,0.01}
\definecolor[named]{ACMRed}{cmyk}{0,0.90,0.86,0}
\definecolor[named]{ACMLightBlue}{cmyk}{0.49,0.01,0,0}
\definecolor[named]{ACMGreen}{cmyk}{0.20,0,1,0.19}
\definecolor[named]{ACMPurple}{cmyk}{0.55,1,0,0.15}
\definecolor[named]{ACMDarkBlue}{cmyk}{1,0.58,0,0.21}

\title{Inferring monotonicity}
\author{Michael Arntzenius}
\institute{University of Birmingham}
\date{S-REPLS 9, 2018}

\newcommand{\R}{\mathbb{R}}
\newcommand{\N}{\mathbb{N}}
\newcommand{\x}{\times}
\newcommand{\G}{\Gamma}
\newcommand{\fn}{\lambda}
\newcommand{\binder}{.~}
\newcommand{\bind}[1]{{#1}\binder}
\newcommand{\fnof}[1]{\fn\bind{#1}}
\newcommand{\den}[1]{\llbracket{#1}\rrbracket}
\newcommand{\bnfeq}{\dblcolon=}

\newcommand{\rulename}{\sffamily\scshape}

\newcommand{\mb}[1]{\ensuremath{\textbf{#1}}}
\newcommand{\mi}[1]{\ensuremath{\textit{#1}}}
\newcommand{\kw}[1]{\mb{#1}}

\newcommand{\id}{\textrm{id}}
\newcommand{\op}{\textrm{op}}
\newcommand{\iso}{\ensuremath{\Box}}
\renewcommand{\path}{\ensuremath{\lozenge}}

\newcommand{\idof}{\mathop{\id}}
\newcommand{\opof}{\mathop{\op}}
\newcommand{\isof}{\iso}
\newcommand{\pathof}{\path}


\begin{document}
\maketitle
\Large

\begin{frame}
  \Huge\centering
  %% Which ones are monotone?
  \[\begin{array}{c}
  x + \log x\vspace{.5em}\\
  x - \log x
  \vspace{.5em}\\
  -2x\vspace{.5em}\\
  4
  \end{array}\]

  %% How did we know that so quickly?
  %% Did we do a proof in our head? No!
  %% We used compositional reasoning!
  %%
  %% I want a type system that tells me the first one is monotone and the second
  %% isn't.
\end{frame}

\begin{frame}%{Types for monotonicity}
  \begin{mathpar}
    \begin{array}{llcll}
      \text{types}& A,B &\bnfeq& \R ~~|~~ A \to B
      \\ \text{terms}& M,N &\bnfeq& x ~~|~~ \fnof{x} M ~~|~~ M\;N
      \vspace{1em}\\
      \multicolumn{2}{r}{\den{A}} &\in& \mathbf{Poset}\vspace{.2em}\\
      \multicolumn{2}{r}{\den{\R}} &=& \R\\
      \multicolumn{2}{r}{\den{A \to B}}
      &=& \text{\emph{monotone} maps}~ \den{A} \to \den{B},\\
      &&& \text{ordered pointwise}
    \end{array}
  \end{mathpar}
\end{frame}

\begin{frame}
  \huge
  \[\begin{array}{l}
    (\fnof{x} x + \log x) : \R \to \R\vspace{.5em}\\
    (\fnof{x} x - \log x) : {??} \to \R
  \end{array}\]
\end{frame}

\begin{frame}
  %% this would work better with a diagram.
  %%
  %% I should explain that \iso is a comonad,
  %% axioms: □A -> A; □A → □□A
  %% but most modal type theories need further annotations
  %%
  %% explain Pfenning-Davies annotations here?

  Let $\isof{A}$ be $A$, ordered \emph{discretely}:
  \[x \le y : {\isof{A}} \iff x = y \]
  \vspace{0pt}

  Then {\color{blue} $f : \isof{A} \to B$ is monotone} iff
  \[x = y \implies f(x) \le f(y) \]
  i.e. \textbf{always}!
\end{frame}

\begin{frame}
  \huge
  \[\begin{array}{l}
    (\fnof{x} x + \log x) : \R \to \R\vspace{.5em}\\
    (\fnof{x} x - \log x) : {\color{Green} \isof{\R}} \to \R
  \end{array}\]
\end{frame}

\begin{frame}
  \iso{} is a \emph{comonad} or \emph{necessity modality}:

  \[
  \begin{array}{rcl}
    \isof A &\to& A\\
    \isof A &\to& \isof{\isof A}\\
    A \to B &\implies& \isof{A} \to \isof{B}
  \end{array}
  \]
  \vspace{0pt}

  Pfenning \& Davies gave typing rules for these in \\
  \emph{A Judgmental Reconstruction of Modal Logic}!
\end{frame}

\begin{frame}
  \LARGE
  \[
  \begin{array}{l}
    {\color{Red} (\fnof{x} x - \log x) : \isof{\R} \to \R}
    \vspace{.5em}\\
    {(\fnof{x} \kw{let}~\kw{box}~u = x~\kw{in}~
      u - \kw{box}\,(\log u)) : \isof{\R} \to \R}
  \end{array}\]
  \vspace{1em}

  \pause\centering\Huge\sffamily :~(
\end{frame}

\begin{frame}{Get rid of pesky annotations in three easy steps!}
  \begin{enumerate}
  \item Exploit features of \mb{Poset}.
  \item Propagate variable usage modes bottom-up.
  \item Use moded subtyping for implicit coercion.
  \end{enumerate}
\end{frame}


%% ---- Exploiting Poset ---
%% Include this once we need to explain \path.
\begin{frame}{1. Exploiting \textbf{Poset}}
  \LARGE%\vspace{-1em}
  %% Which ones are monotone?
  \[\begin{array}{lcl}
  \fnof{x} x + \log x &:& \R \to \R\vspace{.5em}\\
  \fnof{x} x - \log x &:& \isof{\R} \to \R \vspace{.5em}\\
  \fnof{x} {-2}x
  &:& {\alt<2->{\opof{\R}}{\color{Orange}????\hspace{2.9pt}}} \to \R
  \vspace{.5em}\\
  \fnof{x} 4
  &:& {\alt<3->{\color{Green}\pathof{\R}}{\color{Orange}???\hspace{1.3pt}}} \to \R
  \end{array}\]
\end{frame}

\begin{frame}{1. Exploiting \textbf{Poset}: \path}
  $\pathof{A}$ \emph{identifies} weakly connected elements in $A$:
  \[ x \le y \vee y \le x : A \implies x = y : \pathof{A} \]
  \vspace{0pt}

  \pause
  \textbf{Theorem:}
  \[ f : \pathof{A} \to B \iff f : A \to \isof{B} \]
\end{frame}

%% Maybe I need another frame here?


%% ---- 2. PROPAGATING VARIABLE MODES ----
\begin{frame}{2. Propagate variable modes bottom-up}
  
\end{frame}


%% %% Pfenning-Davies vs tonal intro/elim rules.
%% \newcommand{\m}[1]{\left[#1\right]}
%% \newcommand{\disc}[1]{\m{\iso}{#1}}

%% \begin{frame}
%%   \Large
%%   \vspace{-1em}
%%   \begin{mathpar}
%%     \begin{array}{rcl}
%%       T &\bnfeq& 1 ~~|~~ \iso\\
%%       \G &\bnfeq& \varepsilon ~~|~~ \G, x : \m{T}{A}
%%     \end{array}
%%     \\
%%     \infer[\rulename Pfenning-Davies]
%%           {\{x : \m{\iso}{A} \in \G\} \vdash M : A}
%%           {\G \vdash \kw{box}~M : \iso{A}}

%%     \pause
%%     \infer[\rulename Ours]{\G \vdash M : A}{\disc{\G} \vdash M : \isof{A}}
%%   \end{mathpar}
%% \end{frame}

%% \begin{frame}\large
%%   \begin{mathpar}
%%     \infer[PfD \iso-intro]
%%           {\{\disc{A} \in \G\} \vdash A}
%%           {\G \vdash \isof{A}}

%%     \infer[PfD \iso-elim]
%%           {\G \vdash \isof{A} \\ \G, \disc{A} \vdash B}
%%           {\G \vdash B}
%%   \end{mathpar}

%%   \begin{mathpar}
%%     \infer[\iso-intro]{\G \vdash A}{\disc{\G} \vdash \isof{A}}

%%     \infer[\iso-elim]{\G \vdash \isof{A}}{\m{\path}{\G} \vdash A}
%%   \end{mathpar}
%% \end{frame}

%% %% The full Pfenning-Davies rules
%% \begin{frame}{Propagate variable usage bottom-up}
%%   \vspace{-1em}
%%   \begin{mathpar}
%%     \G \bnfeq \varepsilon ~~|~~ \G, x : A ~~|~~ \G, x : \disc{A}
%%     \\
%%     \infer{\{x : \disc{A} ~|~ x : \disc{A} \in \G\} \vdash M : A}
%%           {\G \vdash \kw{box}~M : \iso{A}}
       
%%     \infer
%%         {\G \vdash M : \isof{A} \\
%%           \G, x : \disc{A} \vdash N : B}
%%         {\G \vdash \kw{let}~\kw{box}~x = M ~\kw{in}~ N : B}
%%   \end{mathpar}
%% \end{frame}


%% %% ---- BIDIRECTIONAL INFERENCE ----
%% \newcommand{\inp}{\color{ACMPurple}}
%% \newcommand{\outp}{\color{Cyan}}
%% \newcommand{\inpu}[1]{{\inp{#1}^-}}
%% \newcommand{\outpu}[1]{{\outp{#1}^+}}
%% \newcommand{\checks}[3]{{\inp#1} \vdash {\inp#2} ~\mathbf{checks}~ {\inp#3}}
%% \newcommand{\infers}[3]{{\inp#1} \vdash {\inp#2} ~\mathbf{infers}~ {\outp#3}}

%% \begin{frame}{Input/output modes}
%%   %% Talk about algorithmic interpretation.
%%   \[
%%   \begin{array}{l@{\hskip 2em}l}
%%     \text{Na\"ive} & \inpu{\G} \vdash \inpu{M} : \inpu{A}
%%     \vspace{1em}\\\pause
%%     \text{Hindley-Milner} & \inpu{\G} \vdash \inpu{M} : \outpu{A}
%%     \vspace{1em}\\\pause
%%     \multirow{2}{*}{\text{Bidirectional}}
%%     &
%%     \checks{\inpu{\G}}{\inpu{m}}{\inpu{A}}\\
%%     & \infers{\inpu{\G}}{\inpu{e}}{\outpu{A}}
%%   \end{array}
%%   \]
%% \end{frame}

%% \begin{frame}{Bidirectional type inference}
%%   %% TODO: I should talk about input/output moding here!
%%   \[
%%   \begin{array}{llcl}
%%     \text{checking terms} & m &\bnfeq& \fnof{x} m ~~|~~ e\\
%%     \text{inferring terms} & e &\bnfeq& x ~~|~~ e\;m ~~|~~ (m : A)
%%   \end{array}
%%   \]

%%   %% I should not put all of these on one slide. I should probably ignore some.
%%   %% I only have ~25 mins.
%%   \begin{mathpar}
%%     \infer{\infers{\G}{e}{A \to B} \\ \checks{\G}{m}{A}}
%%           {\infers{\G}{e\;m}{B}}
%%     %% \infer{\checks{\G, x : A}{m}{B}}{\checks{\G}{\fnof{x} m}{A \to B}}
%%     %% \infer{\infers{\G}{e}{B}}{\checks{\G}{e}{B}}
%%     %% \infer{ }{\infers{\G,x : A}{x}{A}}
%%     %% \infer{\checks{\G}{m}{A}}{\infers{\G}{m : A}{A}}
%%   \end{mathpar}
%% \end{frame}


\end{document}
