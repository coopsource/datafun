\documentclass[a5,libertine]{rntz}

%% \documentclass{article}
%% \usepackage{amsmath,amssymb,amsthm}
%% \usepackage{url,hyperref}
%% \usepackage[a4paper,margin=2cm]{geometry}
%% \theoremstyle{plain}
%% \newtheorem{theorem}{Theorem}
%% \newtheorem{corollary}{Corollary}
%% \newtheorem{lemma}{Lemma}
%% \theoremstyle{definition}
%% \newtheorem{definition}{Definition}

%% \usepackage{anyfontsize}
\usepackage{eulervm}              % use Euler math font.

\newcommand{\ms}[1]{\ensuremath{\mathsf{#1}}}
\newcommand{\mb}[1]{\ensuremath{\mathbf{#1}}}
\newcommand{\mi}[1]{\ensuremath{\mathit{#1}}}
\newcommand{\mc}[1]{\ensuremath{\mathcal{#1}}}

\newcommand{\todo}[1]{\noindent{\color{red}#1}}

\newcommand{\N}{\mathbb{N}}
\newcommand{\fn}{\lambda}
\newcommand{\binder}{.\,}
\newcommand{\bind}[1]{{#1}\binder}
\newcommand{\sub}[1]{\;\{{#1}\}}
\newcommand{\Disc}[1]{\square{#1}}

\newcommand{\fix}{\ms{fix}}
\newcommand{\dv}{\delta}

%% TODO: subtitle support in rntz.cls.
\title{$\dv(\fix\;f) = \fix\;(\dv{f} \; (\fix\;f))$
\\
\large or, Static differentiation of monotone fixed points}
\author{Michael Robert Arntzenius}
\date{28 May 2017}

\begin{document}

\maketitle

\begin{center}
  \large
  \textbf{Slogan:}\\
  \it The derivative of a fixed point\\
  is the fixed point of its derivative.
  %% \it A function's fixed point's derivative\\
  %% is the function's derivative's fixed point.
  %% \it The derivative of a function's fixed point\\
  %% is the fixed point of its derivative.
\end{center}

\section{Preliminaries}

I use capital letters $A, B, P, Q$ to stand for posets. I write $P \to Q$ for
the poset of \emph{monotone} maps from $P$ to $Q$. I write $\Disc{P}$ for the
discrete poset on $P$, where $x \le y \iff x = y$.

Fix a poset $P$ with a least element $\bot : P$ admitting a least-fixed-point
operator $\fix : (P \to P) \to P$ such that $\fix\;f = f^i\;\bot$ for some $i :
\N$.

Let us consider a higher-order language $\mc{L}$ which can, among perhaps many
other things, express monotone maps on $P$. Let's assume we have a theory of
changes for $\mc{L}$, \`a la \href{https://arxiv.org/abs/1312.0658}{``A Theory of
  Changes for Higher Order Languages'' by Cai et al, PLDI 2014}. We wish to
extend $\mc{L}$ with $\fix$, and find a theory of changes for the resulting
$\mc{L}_\fix$. In particular, how do we compute derivatives of fixed points? The
answer is in the title:
\[ \dv(\fix\;f) = \fix\;(\dv{f} \; (\fix\;f)) \]

%% TODO: show derivation of this from d(fix f) = d(f (fix f)) and the function
%% application rule.

The correctness criterion for $\dv$ on an expression $e$ with (without loss of
generality) one free variable $x$ is:
\begin{equation}
  e \sub{x \mapsto x \oplus dx} = e \oplus \dv e
  \label{eqn:correctness}
\end{equation}

This article gives a proof that this holds for the provided definition of
$\dv(\fix\;f)$, namely that:
\[ (\fix\;f) \sub{x \mapsto x \oplus dx} = \fix\;f \oplus \fix\;(\dv f \; (\fix\;f)) \]


\section{Assumptions}

I will assume $\Delta P = P$ and make use of this implicitly. This is a fairly
strong assumption, but is true for seminaive Datalog (where $P$ is roughly
``finite sets of facts'' or ``relations'') and in Datafun. I do not provide
definitions of $\oplus$, $\ominus$ etc for $P$; I assume they are already a part
of $\mc{L}$'s theory of changes. I will also assume the following lemmas:

\begin{lemma}[Bottom changes]
  \label{lem:bot-change}
  $\bot : \Delta P$ is a valid change to any $x : P$, and $x \oplus \bot = x$.
\end{lemma}

NB. Not all values $dx : \Delta{A}$ are necessarily \emph{valid} changes to a
given $x : A$. For example, $-3$ is an invalid change to the natural number $0$,
because $0 + -3$ is not a natural number. (The Cai et al paper formalizes this
point using a dependently-typed theory of changes. I will argue a bit more
informally.)

\begin{lemma}[Function changes]
  \label{lem:fun-change}
  If $df$ is a valid change to $f : A \to B$ and $dx$ is a valid change to $x :
  A$, then:
  \begin{enumerate}
  \item $df\; x\; dx$ is a valid change to $f\;x$.
  \item \( (f \oplus df)\;(x \oplus dx) = f\;x \oplus df\;x\;dx \)
  \end{enumerate}
\end{lemma}

\begin{lemma}[Curiously specific]  \label{lem:curiously-specific}
  For $f : P \to P$, we have $f \le f \oplus \dv f$.
\end{lemma}

The substance of this ``curiously specific'' lemma is that all derivatives of
functions which we take fixed points of must be \emph{increasing} changes. In
Datafun this is true universally: all changes are increases.


\section{Proof of correctness}

Consider some expression $f : P \to P$ in one free variable $x$, whose
derivative is $\dv f : \Disc{P} \to \Delta P \to \Delta P$. We assume $\dv f$ is
a correct derivative, and therefore by equation \ref{eqn:correctness}:
\[
  (\fix\;f)\sub{x\mapsto x \oplus dx}
  = \fix\;(f\sub{x\mapsto x \oplus dx})
  = \fix\;(f \oplus \dv f)
\]

Thus it suffices to show that:
\[ \fix\;(f \oplus \dv f) = \fix\;f \oplus \fix\;(\dv f\; (\fix\;f)) \]

To do this, we will need the corollary of one crucial lemma:


\begin{lemma}[Iterated changes]
  \label{lem:crux} If $df$ is a valid change to $f$ and $dx$ is a valid change to $x : P$, then for any $i : \N$,
  \[
  (f\oplus df)^i \;(x \oplus dx)
  = f^i\;x \oplus
  (df\;(f^{i-1}\;x) \circ df\;(f^{i-2}\;x) \circ ...
  \circ df\;(f\;x) \circ df\;x)\; dx
  \]
\end{lemma}

\begin{proof}
  Induct on $i$, applying Lemma \ref{lem:fun-change}. The base case is $x \oplus
  dx = x \oplus dx$. The inductive case is:
  \[
  \begin{array}{cll}
     & (f \oplus df)^{i+1}\; (x \oplus dx)\\
    =& (f \oplus df)\; (f^i\;x \oplus
    (df \;(f^{i-1}\;x) \circ ... \circ df\;x)\;dx)
    & \text{by our inductive hypothesis}\\
    =& f^{i+1}\;x \oplus df\; (f^i\;x) \;
    ((df \;(f^{i-1}\;x) \circ ... \circ df\;x)\;dx)
    & \text{by Lemma \ref{lem:fun-change}}\\
    =& f^{i+1}\;x \oplus
    (df\;(f^i\;x) \circ df\;(f^{i-1}\;x) \circ ... \circ df\;x)\;dx
  \end{array}
  \]
\end{proof}

\begin{theorem}
  \label{thm:crux}
  For any $i : \N$ and any valid change $e : \Delta{P}$ to $\fix\;f$,
  \[
  (f\oplus\dv f)^i \;(\fix\;f \oplus e)
  = \fix\;f \oplus (\dv f \; (\fix\;f))^i \;e
  \]
\end{theorem}
\begin{proof}
  Apply Lemma \ref{lem:crux} with $x = \fix\;f$ and $dx = e$, noting that
  $f^j\;(\fix\;f) = \fix\;f$ for any $j$.
\end{proof}


%% \begin{lemma}[Crux]
%%   \label{lem:crux}
%%   For any $i : \N$ and any valid change $e : \Delta{P}$ to $\fix\;f$,
%%   \[
%%   (f\oplus\dv f)^i \;(\fix\;f \oplus e)
%%   = \fix\;f \oplus (\dv f \; (\fix\;f))^i \;e
%%   \]
%% \end{lemma}

%% I'd like to be able to give a plain-English explanation for the intuition behind
%% this lemma, but I don't have one. If you do, email me at
%% \href{mailto:daekharel@gmail.com}{daekharel@gmail.com}.

%% \begin{proof}
%%   Induct on $i$. When $i = 0$, this becomes trivial: $\fix\;f\oplus e = \fix\;f
%%   \oplus e$. Otherwise, we'll assume it holds for some $i$ (and for any valid
%%   $e$) and prove it for $i+1$:

%%   \[\begin{array}{cll}
%%    & (f \oplus \dv f)^{i+1} \; (\fix\;f \oplus e)\\
%%   =& (f \oplus \dv f)^i \; ((f \oplus \dv f) \; (\fix\;f \oplus e))\\
%%   =& (f \oplus \dv f)^i \; (f\;(\fix\;f) \oplus \dv f \;(\fix\;f) \;e)
%%   & \text{by Lemma \ref{lem:fun-change}, since $\dv f$ and $e$ are valid}\\
%%   =& (f \oplus \dv f)^i \; (\fix\;f \oplus \dv f \;(\fix\;f) \;e)
%%   & \text{b/c $\fix\;f$ is a fixed point}\\
%%   =& \fix\;f \oplus (\dv f \;(fix\;f))^i \;(\dv f \;(\fix\;f) \;e)
%%   & \text{by IH, letting $dx = \dv f \;(\fix\;f) \;e$ (*)}\\
%%   =& \fix\;f \oplus (\dv f\;(\fix\;f))^{i+1}\;e
%%   \end{array}\]

%%   Which is what we wished to show. (At the point marked (*), there is a slight
%%   omission: by Lemma \ref{lem:fun-change} and the validity of $\dv f$ and $e$ we
%%   know $\dv f\; (\fix\;f) \;e$ is a valid change to $f\;(\fix\;f) = \fix\;f$.
%%   Thus we can apply our IH.)
%% \end{proof}


We'll also need a simple but cute lemma about monotone fixed points in general:

\begin{lemma}[Jason Reed's lemma] \label{lem:jcreed}
  For any $f : P \to P$, if $a \le \fix\;f$ and $\fix\;f = f^i\;\bot$ then
  $f^i\;a = \fix\;f$.
\end{lemma}
\begin{proof}
  Since $\bot \le a$, by monotonicity $\fix\;f = f^i\;\bot \le f^i\;a$. Since $a
  \le \fix\;f$, by monotonicity $f^i\;a \le f^i\;(\fix\;f) = \fix\;f$. So by
  antisymmetry $f^i\;a = \fix\;f$.\footnote{This is a special-case of a more general lemma, which you can find at \url{https://mathoverflow.net/questions/269117/random-iteration-of-a-set-of-monotone-maps-until-fixed-point}.}
\end{proof}

Armed with these, we can prove our main result:

\begin{theorem}[Correctness of $\dv(\fix\;f)$]
  \[ \fix\;(f \oplus \dv f) = \fix\;f \oplus \fix\;(\dv f\; (\fix\;f)) \]
\end{theorem}

\begin{proof}
  Pick $i$ such that:
  \begin{eqnarray*}
    \fix\;(f \oplus \delta f) &=& (f \oplus \dv f)^i \;\bot\\
    \fix\;(\dv f \;(\fix\;f)) &=& (\dv f\;(\fix\;f))^i \;\bot
  \end{eqnarray*}

  We can pick such an $i$ because it's always the case that $\fix\;f = f^i \bot$
  for some ``iteration count'' $i$; and since $f^i\;\bot$ is a fixed point, we
  can increase $i$ without changing its value. So just take the larger of the
  ``iteration counts'' for $f \oplus \dv f$ and $\dv f\;(\fix\;f)$.

  Then:
  \[
  \begin{array}{cll}
     & \fix\;(f \oplus \dv f)\\
    =& (f \oplus \dv f)^i \; (\fix\;f)
    & \text{by Reed's Lemma \ref{lem:jcreed}, since $\fix\;f \le \fix\;(f \oplus \dv f)$ (*)}\\
    =& (f \oplus \dv f)^i \; (\fix\;f \oplus \bot)
    & \text{by Lemma \ref{lem:bot-change}}\\
    =& \fix\;f \oplus (\dv f \;(\fix\;f))^i \;\bot
    & \text{by Theorem~\ref{thm:crux}, as $\dv f$ is valid \& $\bot$ is valid by Lemma \ref{lem:bot-change}}\\
    =& \fix\;f \oplus \fix\;(\dv f\;(\fix\;f))
  \end{array}
  \]

  Which is what we wished to show.

  %% TODO: doesn't this show that we don't need \bot to be the zero-change, we
  %% just need there to be *a* zero change? And we need it to be the least
  %% change, so that fix : (\Delta P -> \Delta P) -> \Delta P can be computed in
  %% terms of it.

  At (*), $\fix\; f \le \fix\; (f \oplus \dv f)$ follows from monotonicity of
  $\fix$ applied to $f \le f \oplus \dv f$ (Lemma \ref{lem:curiously-specific}).

\end{proof}

\end{document}
