\documentclass{sigplanconf}

% The following \documentclass options may be useful:

% preprint      Remove this option only once the paper is in final form.
% 10pt          To set in 10-point type instead of 9-point.
% 11pt          To set in 11-point type instead of 9-point.
% numbers       To obtain numeric citation style instead of author/year.

\usepackage[utf8]{inputenc}
%% 2018-04-30: This used to be \usepackage[T1]{fontenc}, but for some reason
%% that now causes a font scaling error ("auto expansion is only possible with
%% scalable fonts").
%\usepackage{fontenc}
\usepackage{microtype}          % minor typographical improvements, apparently
\usepackage{flushend}           % not sure what this is for

\usepackage{datafun}
\renewcommand{\todo}[1]{}


%% ---------- Setup ----------
\begin{document}
\toappear{}
\special{papersize=8.5in,11in}
\setlength{\pdfpageheight}{\paperheight}
\setlength{\pdfpagewidth}{\paperwidth}

\conferenceinfo{ICFP '16}{18--24 September, 2016, Nara, Nara, Japan}
\copyrightyear{2016}
\copyrightdata{978-1-nnnn-nnnn-n/yy/mm}
\copyrightdoi{nnnnnnn.nnnnnnn}

% Uncomment the publication rights you want to use.
%\publicationrights{transferred}
%\publicationrights{licensed}     % this is the default
%\publicationrights{author-pays}


%% ---------- The title ----------
% These are ignored unless 'preprint' option specified.
\titlebanner{preprint}
\preprintfooter{Datafun: A Functional Datalog (PREPRINT)}

\title{Datafun: A Functional Datalog}
\subtitle{}

\authorinfo{Michael Arntzenius\and Neelakantan R. Krishnaswami}
           {University of Birmingham (UK)}
           {\{daekharel, neelakantan.krishnaswami\}@gmail.com}

\maketitle


%% ---------- The abstract ----------
\begin{abstract}
  Datalog may be considered either an unusually powerful query language or a
  carefully limited logic programming language. Datalog is declarative,
  expressive, and optimizable, and has been applied successfully in a wide
  variety of problem domains. However, most use-cases require extending Datalog
  in an application-specific manner. In this paper we define Datafun, an
  analogue of Datalog supporting higher-order functional programming. The key
  idea is to \emph{track monotonicity with types}.
\end{abstract}

\category{F.3.2}{Logics and Meanings of Programs}{Semantics of Programming Languages}
%
% % general terms are not compulsory anymore,
% % you may leave them out
% \terms
% term1, term2
%
\keywords Prolog, Datalog, logic programming, functional programming, domain-specific languages,
type theory, denotational semantics, operational semantics, adjoint
logic


%% ---------- Paper body ----------
%% Section 1: Introduction
\input{introduction}

%% Section 2: Datafun, informally
\input{informally}

%% Section 3: examples
\input{examples}

%% Section 4: Typing rules
\input{typing-rules}

%% Section 5: Denotational semantics
\input{denotational-semantics}

%% Section 6: Operational semantics
\input{operational-semantics}

%% Section 7: Datafun vs Datalog
\input{datalog-comparison}

%% Section 8: Implementation
\input{implementation}

%% Section (omitted): Tradeoffs, etc.
\input{tradeoffs}

%% Section 9: Related & future work
\input{related-future-work}


%% ---------- End matter ----------

\acks

Thank you to:
\begin{itemize}
\item Chris Martens, for helpful feedback and discussion on connections to logic
  programming.
\item The anonymous ICFP reviewers, for pointing out areas which were previously
  unclear or incomplete.
\end{itemize}

% We recommend abbrvnat bibliography style.
\bibliographystyle{abbrvnat}
\bibliography{datafun}

%% \appendix
%% \section{Appendix Title}

%% This is the text of the appendix, if you need one.


\end{document}
